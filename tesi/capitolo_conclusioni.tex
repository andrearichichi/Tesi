\chapter{Conclusioni}
\label{cha:conclusioni}

In questo capitolo si espone l'analisi degli obiettivi raggiunti e le funzionalità di VolleyVisionAI che ho sviluppato e di come il percorso formativo seguito da studente ha influenzato il processo di progettazione e realizzazione, analizzando lo stato attuale del progetto e concludendo con riflessioni personali sul lavoro svolto e sui risultati ottenuti.

\section{Raggiungimento degli obiettivi}

VolleyVisionAI ha raggiunto gli obiettivi prefissati con successo, completando tutte le fasi necessarie alla realizzazione della Web App. 
In particolare, dopo un'attenta analisi di mercato, ho definito in fase di progettazione i requisiti chiave da implementare. Successivamente ho creato un prototipo dell'interfaccia con Figma, per poi arrivare alla fase di sviluppo vera e propria. Attualmente l'app è funzionante nella sua versione Beta reperibile alla repository GitHub di FBK come VolleyVisionAI \cite{VolleyVisionAI}.






\subsection{Funzionalità della Web App}

Come descritto nel capitolo \ref{cha:analisi_progettazione}, le funzionalità sviluppate permettono all'utente di: 
\begin{itemize}
    \item creare, importare ed esportare progetti di analisi video personali.
    \item creare \textit{shortcut} personalizzati specificando atleta, azione e comando rapido da tastiera.
    \item etichettare eventi specifici durante l'analisi video utilizzando gli shortcut creati in precedenza.
    \item raggruppare gli eventi etichettati per giocatore o azione. 
    \item filtrare gli eventi etichettati tramite una barra di ricerca.
    \item assegnare valutazioni a ciascun evento etichettato.
    \item disegnare e fare annotazioni direttamente sul video tramite una \textit{dashboard} dedicata. 
    \item richiedere all'app di generare un Videoclip contenente le azioni etichettate in precedenza.
    \item utilizzare l'analisi AI per visualizzare la traiettoria della palla durante le fasi di gioco.
    \item utilizzare l'analisi AI per identificare automaticamente i singoli giocatori in campo.
\end{itemize}


\section{Sviluppi Futuri}

VolleyVisionAI è ancora in una fase Beta e sono numerose le aree di sviluppo che potenzialmente possono rendere l'app ancora più completa e funzionale. In particolare, alcuni possibili sviluppi futuri sono descritti dalle seguenti macro-categorie:

\begin{itemize}
    \item \textit{Gestione utenti}: Sarà necessario implementare un sistema di registrazione e autenticazione per consentire agli utenti di avere il proprio account personale sulla piattaforma. Questo permetterà una gestione dei progetti molto più flessibile e dinamica.
    \item \textit{Modelli Computer Vision}: Come specificato nel nome della Web App, la direzione principale è l'implementazione di ulteriori funzionalità AI. In particolare, lo sviluppo di un modello di Computer Vision per il riconoscimento automatico delle azioni, evitando così il processo di \textit{Event Tagging} manuale. 
    \item \textit{Open Source}: In accordo con il tutor aziendale Maurizio Napolitano, è stata presa in considerazione la possibilità di rendere il progetto VolleyVisionAI Open Source. Questo permetterebbe di velocizzare la crescita della Web App e, allo stesso tempo, diffonderla il più possibile.
    \item \textit{Deployment}: Per rendere l'app accessibile a molti più utenti, sarà necessario implementare un sistema di deployment. L'idea è quella di utilizzare inizialmente servizi di cloud hosting e in particolare l'utilizzo di \textit{Vercel} per il Front-End e \textit{Heroku} per il Back-End.
\end{itemize}

\noindent Durante la fase di progettazione tecnica è stata definita un'architettura estremamente scalabile composta da due server: Front-End e Back-End. Anche se il deploy verrà effettuato utilizzando servizi di cloud hosting, a cui verrà delegata in gran parte la gestione delle risorse, le tecnologie utilizzate sono state scelte per garantire un'ottima scalabilità. In particolare \textit{React} e \textit{FastAPI} permettono di creare applicazioni modulari e facilmente estendibili, garantendo un'ottima manutenibilità del codice.

\section{Formazione}

Per la creazione dell'app VolleyVisionAI, lo studio delle tecnologie utilizzate in fase di sviluppo è stato individuale. Il corso universitario di \textit{Ingegneria del Software} mi ha aiutato in fase di analisi e progettazione a definire i requisiti funzionali e non funzionali della Web App. Inoltre, il corso di EyeStudios Academy su \textit{Figma} mi ha permesso di implementare un prototipo interattivo dell'interfaccia utente prima di passare alla fase di sviluppo tecnico. 
Grazie all'esperienza di tirocinio formativo presso la \textit{Fondazione Bruno Kessler} ho potuto approfondire tutte le fasi dello sviluppo web. In particolare, ho acquisito competenze nell'utilizzo di \textit{React} e \textit{JavaScript}, che sono state fondamentali per la realizzazione della Web App. 
Anche il corso di \textit{Introduction to Machine Learning} tenuto dalla professoressa Elisa Ricci mi ha aiutato nel comprendere i concetti principali di questo settore, che poi ho utilizzato durante lo sviluppo della Web App. 

\section{Conclusioni Personali}

Il lavoro svolto per realizzare VolleyVisionAI mi ha fornito un'occasione preziosa per applicare e approfondire le conoscenze acquisite durante il percorso universitario. 
Lo sviluppo di un progetto pratico sarà sicuramente vantaggioso per i futuri impegni lavorativi e rappresenterà un valore aggiunto al proprio portfolio personale.
Questo progetto, insieme all'esperienza di tirocinio, ha consolidato la mia passione per lo sviluppo web e per l'intelligenza artificiale, rendendomi consapevole del percorso formativo che intendo intraprendere in futuro.




