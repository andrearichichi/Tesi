\chapter{Introduzione}
\label{cha:introduzione}


L'obiettivo di questo progetto è sviluppare una Web App per l'analisi video nel volley dedicata ad allenatori e atleti. L'idea nasce in seguito ad alcune ricerche di mercato che hanno evidenziato l'assenza di soluzioni \textit{user-friendly} gratuite, per l'analisi video nello sport della pallavolo, adatte ad utenti amatoriali. L'applicazione, rispetto ad altre soluzioni già presenti sul mercato, ha infatti lo scopo principale di avere una \textit{User Interface} estremamente semplice ed intuitiva, offrendo allo stesso tempo un riuso gratuito. Questo permette di garantire una \textit{User Experience} di alto livello, qualunque siano le conoscenze tecniche dell'utente finale.

Le competenze accademiche e lavorative acquisite durante il percorso di laurea Triennale in Informatica a Trento, sono risultate fondamentali in fase di progettazione e sviluppo. Il corso online tenuto da EyeStudios Academy \cite{EyeStudios-Academy} su \textit{Figma} ha fornito l'esperienza necessaria per realizzare un prototipo interattivo dell'app, velocizzando così il processo di sviluppo.

L'esperienza di tirocinio presso la \textit{Fondazione Bruno Kessler} ha inoltre contribuito a consolidare la conoscenza della libreria \textit{React} e del linguaggio \textit{JavaScript}, ampiamente utilizzati nello lo sviluppo della Web App.

Il corso di Ingegneria del Software ha aiutato a definire, in fase di analisi e progettazione, le funzionalità dell'app individuando requisiti funzionali e non funzionali.

Il resto dell'elaborato prevede questi capitoli:

\begin{itemize}
    \item \textbf{Stato dell'Arte}: dove viene introdotto il concetto di \textit{Sport4.0} con particolare attenzione al mondo della pallavolo e presentate le tecnologie e le soluzioni attualmente presenti sul mercato.
    \item \textbf{Analisi e Progettazione}: presenta le fasi di analisi e progettazione della Web App con le definizioni dell'interfaccia e requisiti.
    \item \textbf{Implementazione e Sviluppo}: descrive lo sviluppo pratico dell'applicazione mediante la rappresentazione dell'architettura di sistema e le tecnologie utilizzate.  
    \item \textbf{Evaluation}: qui vengono riportati i risultati relativi ad un test di usabilità effettuato da parte degli utenti. Sono, inoltre, presentati i loro feedback ed i miglioramenti apportati di conseguenza.
    \item \textbf{Conclusioni}: riassume le caratteristiche dell'app, ripercorrendo gli obiettivi raggiunti e conclude valutando possibili sviluppi futuri e riflessioni personali.   
\end{itemize}




La tesi descrive i dettagli del progetto, iniziando dall'analisi del contesto e delle esigenze degli utenti, continuando con la progettazione e lo sviluppo della Web App, fino alla valutazione dei risultati ottenuti. Il tutto è debitamente illustrato, quando necessario, rendendo l'elaborato maggiormente comprensibile.


