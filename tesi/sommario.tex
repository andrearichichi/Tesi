\chapter*{Abstract}
\label{Abstract}
\addcontentsline{toc}{chapter}{Abstract}

L'elaborato presenta lo sviluppo e l'implementazione di \textit{VolleyVisionAI}, una Web App progettata per l'analisi video nello sport della pallavolo. L'obiettivo principale è quello di colmare il vuoto di mercato tra gli strumenti professionali, spesso inaccessibili, e l'esigenza di un'applicazione intuitiva e semplice rivolta ad un'utenza amatoriale. 

\textit{VolleyVisionAI} presenta due modalità di funzionamento: la modalità manuale, che permette agli utenti di creare, importare ed esportare progetti di analisi video, effettuare \textit{Event Tagging} e generare videoclip personalizzati; e la modalità AI, che utilizza modelli di \textit{Computer Vision} per tracciare la palla e riconoscere i giocatori in campo. Queste funzionalità sono state implementate sviluppando un'architettura che garantisce alte prestazioni e scalabilità, con \textit{React.js} per il front-end e \textit{FastAPI} per il back-end.

Il processo di sviluppo ha compreso l'analisi dei requisiti, la creazione di un prototipo iniziale con \textit{Figma}, e lo sviluppo tecnico di interfaccia utente in \textit{JavaScript} e funzionalità back-end in \textit{Python}. Queste ultime comprendono l'analisi tramite AI per la quale è stato integrato il modello di riconoscimento \textit{YOLOv9} e la libreria \textit{OpenCV}. 

I test di usabilità effettuati hanno confermato l'efficacia dell'applicazione e l'intuitività della sua interfaccia. Inoltre i feedback ricevuti hanno permesso di migliorare ulteriormente la \textit{user-experience} e offrire spunti per sviluppi futuri dell'app. 
La versione beta di \textit{VolleyVisionAI} è attualmente operativa e ha raggiunto pienamente gli obiettivi prefissati. I risultati ottenuti offrono nuove opportunità per il futuro, tra cui l'aggiunta di nuove funzionalità nella modalità AI e l'estensione dell'accessibilità attraverso un sistema di \textit{Deployment}.
